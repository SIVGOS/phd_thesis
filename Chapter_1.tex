\chapter{Introduction}
\label{chap:chap1}
% This chapter should contain the brief introduction of the thesis, literature survey, motivation, problem formulation, contribution and organization of the thesis.
%\minitoc
\def\baselinestretch{1.66}
This chapter starts with some background information on the trends of soft-computational approaches to the design of microstrip antennas and arrays. A set of problems are identified through an extensive literature survey and the orientation of the thesis is outlined in the chapter.

\section{Background} \label{c1sec_background}
Antenna design technologies have significantly advanced in the last decade. Computer aided methods are emerging as a solution to build more complex patch antennas and arrays with better efficiency and performance. Soft-computational approaches in the design of antennas and arrays have unleashed highly optimized geometries that cannot be designed by humans. This advancement has contributed towards the development of compact and reliable communication gadgets that have helped in connecting the whole world.

\subsection{Soft-Computational Optimization Techniques}
Soft-computational optimization techniques, also known as \emph{derivative-free} optimization techniques, have revolutionized the approach for solving many engineering problems. These algorithms provide a metaheuristic approach to solve problems for which it is difficult to obtain derivative information of the objective function over the solution space. These algorithms are iterative in nature and use some stochastic method to search the solution space to obtain an optimum solution \cite {softCompBook}. These algorithms don't guarantee global maxima, and based on problem the performance of the algorithms vary \cite{compCAD4Arry}.

Most of the soft-computational optimization algorithms are derived by mimicking nature. One of the most widely used example of these algorithms is the genetic algorithm (GA). It mimics the Darwin's theory of natural selection in the evolution of a species. Differential evolution (DE) is another popular algorithm that is inspired by the theory of evolution. Particle swarm optimization (PSO), another popular algorithm, mimics swarm intelligence that include the behavior of ant colonies, bird flocking, animal herding, bacterial growth, fish schooling and microbial intelligence. Apart from these three, there are many soft-computational optimization algorithms reported. Some of these algorithms include:
\begin{itemize}
\item Teaching learning based optimization (TLBO) algorithm \cite{arraySynth3}
\item Chicken swarm optimization (CSO) algorithm \cite{arraySynth4}
\item Spider monkey optimization (SMO) algorithm \cite{arrayThin1}
\item Binary butterfly mating optimization (BBMO) algorithm \cite{arrayThin2}
\item Cuckoo search (CS) algorithm \cite{CuckooSerach}
\item Invasive Weed Optimization (IWO) \cite{InvasiveWeed}
\end{itemize}

\subsection{Design of Microstrip Antennas}
Microstrip antennas are low profile, light, conformal and durable antennas which are ideal for use in portable and compact communication devices. Microstrip antennas are usually fabricated on a PCB board which can be integrated with other electronic circuits easily. There are three approaches for the analysis and working of microstrip antennas:
\begin{itemize}
\item Transmission line model
\item Cavity model
\item Full-wave model
\end{itemize}

The transmission line model is least accurate but gives more insights to the designing and working of antennas. The full-wave model, on the other hand, provides the least insights to the designing of an antenna but have the highest accuracy. The cavity model has accuracy between the transmission line model and the full-wave model. It provides some insights into the design and working of the antenna \cite{balanis}.

For a simple rectangular or circular patch, it is easy to design or study an antenna analytically using either a transmission line model or a cavity model. However, for arbitrarily shaped antennas, it is difficult to formulate a transmission line model or a cavity model. This adds up to the challenges in antenna design. Further, these antennas need fine-tuning in order to achieve desired performance at the frequencies of interest. Due to the inaccuracies of the transmission line model and the cavity model, regularly shaped antennas also needs fine-tuning and optimization. This process is time and labor intensive and requires a significant expertise of the domain.

Because of the limitations of the traditional approaches for antenna design, soft-computational approaches are becoming popular. An evolutionary-based algorithm can be used to search the design space and automatically find novel antenna designs that are more effective than would otherwise be developed \cite{cadNASA}. Evolutionary-based approaches in antenna design are largely explored in the recent years. Various soft-computational techniques have been used to design antennas of various sizes and shapes. Sometimes, the algorithms are also tailored to make them more suitable for this purpose.

\subsection{Design of Antenna Arrays}
Antenna arrays are very important in the design of directional communication system. Arrays are inevitable in modern-day communication systems which make use of spatial diversity. The far-field radiation pattern of an array can be electrically steered by adjusting the phase of excitation of each individual element of an array. Similarly, it is possible to avoid interference from a specific direction by generating a null along that direction. A MIMO communication system is a good example of arrays in modern-day communication. Arrays with electrically steerable beams are also used in radars.

A thinned array or a sparse array is an array in which the number of elements of the array is reduced in such a way that the performance in terms of the radiation pattern and gain of the array is not significantly compromised. Sparse arrays help in reducing the size and complexity of the RF circuitry required for exciting each element at a different phase. Design of sparse array is another field in which soft-computational approaches are becoming popular.

Other problem in the array design includes side-lobe level (SLL) reduction and impedance matching. In SLL reduction, the primary objective is to reduce the power of the antenna along the side lobes and maximize the gain of the antenna along the main lobe in the desired direction. The objective of an impedance matching problem in array design is to make sure that the input impedance of each of the elements in an array is properly matched with that of the respective feeding lines. Soft-computational approaches are explored in both of these cases.

\subsection{Problems in Soft-Computational Approaches for Antenna and Array Design}
The common design problems in microstrip antennas include reduction of the electrical size of the antenna, enhancement of gain, enhancement of bandwidth, obtaining desired polarization performance, the design of filter integrated with antennas etc. In this section, some of the popular problems of evolutionary-based approaches in the design of microstrip antennas and arrays are discussed.

Optimization problems in antenna design can be broadly classified into two categories -
\begin{itemize}
\item Continuous problem
\item Binary problems
\end{itemize}

Optimizing various dimensions of an antenna is an example of a continuous problem. In a binary problem, the primary objective is usually to find the presence (0) or absence (1) of metal in a position to find the shape of the patch. In a continuous problem, the topology of the antenna remains fixed. One or more physical dimensions of the antenna are tuned over a continuous range to match requisite performance of the antenna in terms of resonant frequency, gain, bandwidth etc.

The theoretical development of antenna arrays mostly evolve around uniform patterns \cite{classicTheory, phasedArrayHandbook}. With the evolution of 5G and millimeter wave standards, arrays are becoming larger and denser. Reducing the number of elements of a large array without significantly degrading its performance has been an area of research for several decades \cite{thinnedArrayBook}. This problem is called \emph{array thinning problem}. With the evolution of soft-computational optimization techniques, these algorithms are widely in use for such problems.

A typical array thinning problem starts with a full linear or planar array. The algorithm iteratively finds some elements of the array that can be removed from from the array while trying the achieve the pre-defined target. Array thinning problems are thus discrete problems.

\section{Literature Survey} \label{c1sec_litserv}
Printed antennas have made a significant contribution to the evolution of electronic devices and tools to its present state of excellence. Due to their robust and compact architecture, they can be easily interfaced with circuits. Most of the time, printed antennas are fabricated on the same substrate with other electronic circuits of the system. Because of these qualities, printed antennas are able to attract worldwide attention among researchers.

Microstrip antennas are the first printed antennas. The concept of microstrip antennas was first established in 1953 by G. A. Deschamps et al. \cite{mpa00}. Microstrip antennas and arrays were practically implemented in 1970s in a number of contemporary works such as \cite{mpa02} and \cite{mpa01} respectively. The design trends till the beginning of 1980s was surveyed by K. R. Karver et al\cite{mpaSurvTech}. This review work gives a significant amount of insights to the design trends until that period. The number of papers related to microstrip antennas started to grow exponentially over the years after 1980 \cite{mpaHist01}. The design of antenna arrays using microstrip antennas was another trend that evolved almost simultaneously. At that time, microstrip antennas were known for its disadvantage of low gain and bandwidth. This may be a reason why many researchers considered design of compact, rigid, planar arrays using microstrip antennas to enhance the gain. In 1974, a phased array of microstrip antenna elements was proposed by R. F. Munson \cite{txmPhasedArray}.

In the initial phase, most of the works on microstrip antennas were related to feeding mechanisms. In these works there are some simplifying approximations involved for deriving the behavior of the feeding techniques, however, they are easy and they provide useful information regarding impedance, radiation patterns, efficiency, bandwidth of the antennas \cite{mpaReview1992}. The microstrip line feeding and coaxial feeding are popular since 1970s and at present also they are the most commonly used feeding mechanism for microstrip antennas.

Between mid 1970s and early 1980s, for the first time, there were several research works reported the design and modeling of microstrip antennas. The transmission line model and the cavity model were formulated during these period \cite{handbook}. Finally, with the availability of modern computers, full-wave electromagnetic solvers were created. By 1990s there were commercially available full wave electromagnetic solvers. This led to a remarkable increase in the volume of research published in the field of antenna design. Antennas with highly complex geometrical structures can be easily designed and simulated using software applications such as Ansys HFSS{\circledR}, CST Studio{\circledR} etc. \cite{practGuide3D}. Antennas of various sizes and shapes were designed during this period. The analytical modeling and parameterizations of antennas, on the other hand, became less popular. Some of the analytical works published were only focused on providing some insights to the working of the antenna. The designs are mostly validated from simulation and measurement results.

With the evolution of modern computers, soft-computational tools and techniques are becoming popular in all areas of research. Soft-computational techniques are widely reported in the field of antenna design as well. In the remaining part of this section, the whole domain of printed antenna research is divided into three categories. Some recent advancements in each category are reviewed as follows.

\subsection{Design or Synthesis of Printed Antenna Elements}
Nowadays many new full-wave techniques has evolved such as the finite elements method (FEM), finite integration technique (FIT), etc. which are more efficient and offer varying degree of accuracy for different types of structures \cite{practGuide3D}. In \cite{practGuide3D}, various commercially available full-wave solvers are compared. The mathematics behind many of these techniques is available in \cite{handbook} and \cite{numericalbook}. From 1990s till the present time, there have been several parallel trends in the field of microstrip antenna design. Some of these trends include design of slotted antennas and patches of different geometry \cite{HPatch1, uslot1}, design of microstrip antennas with reduced size and increased bandwidth \cite{smallPatch0, BandSize0}, design of antennas with multiple resonant frequencies \cite{dualBandPatch1, dualBandCircPol,fractal1}, design of ultra wide band antennas \cite{slottedUWB, PMA01}, design of metamaterial based antennas with integrated filters \cite{bandnotchCSRR1, bandnotchEBG1}, and design of antennas with minimized RCS \cite{rcs1992, rcs2014, rcs2016}

Due to the availability of the numerical full-wave tools, many new areas in the field of printed array design were also explored. Some of these newly identified problems are design of feeding mechanism for compact array application \cite{CompFeed01, CompFeed02}, reduction of mutual coupling in arrays using metamaterial \cite{mcEBG, mcCSRR, mcFSRR, mimoSRR}, array design for MIMO application \cite{mimoConformal, mimoDualBand}, design of microstrip grids for massive MIMO \cite{mimoSRR, mimoRenonf}, and switched parasitic arrays for electrically steerable beams \cite{spa01,spaMems}.

The search for solution to many new problems using the simulation based approach resulted in some completely new designs. Some of these new design elements explored include \emph{printed slot antennas}, \emph{printed monopole antennas}, \emph{metamaterial based designs} etc. The printed slot antennas are known for yielding high bandwidth at a single resonant frequency \cite{PSA1}. The printed monopole antennas, on the other hand, are widely accepted as good candidates for the design of ultra wide-band (UWB) antennas \cite{PMA01}. The printed slot antenna and the printed monopole antenna are not the same as microstrip antennas. The printed monopole antenna does not have a ground plane.The printed slot antenna is, on the other hand, a complementary structure of the microstrip antenna. Thus, the cavity model does not hold good for these structures. There are some transmission line analysis reported for these antennas, however, the methods are devised only to provide some insights to the operation of the antennas and the performance of the antennas are validated from full-wave analysis \cite{slotAnalysis, pmaCSRR}.

Metamaterials are some artificial structures which can yield negative value of the permittivity and permeability of a structure at some frequencies. There are various metamaterial structures which are used with printed antennas and arrays to improve their performance \cite{mtmReview1, mtmReview2}.

The following are some research works where soft-computational optimization techniques are used for the synthesis of antenna elements:
\begin{itemize}
\item \cite{cadNASA2} is one of the pioneering works where a soft-computational optimization technique was used for synthesis of an antennas was reported by J. Lohn et al. in 2004. In this work, evolutionary based algorithm is used for synthesizing the structure of a wire monopole antenna that has for arms identical to one another for NASA's ST-5 mission. From the research works presented, it is evident that NASA has been using soft-computational algorithms for optimization of antennas and arrays to meet their requirements for various satellites and space missions.
\item G. S. Hornby et al. in 2006 reported a revised algorithm for computer aided synthesis of an antenna for NASA's ST-5 mission \cite{cadNASA}. The design was to compensate for a null in the zenith direction that was not in line with the revised requirements for the mission. An additional penalty function was included in the objective function to achieve the task. The new design improves the performance of the antenna without the unwanted null.
\item In 1997, J. M. Johnson et al. published a tutorial paper on the use of genetic algorithm for electromagnetic application \cite{ga_em}. In this paper, a comprehensive study is presented on various aspects of antenna optimization and how genetic algorithms may be applied for this purpose. The paper includes the selection of fitness functions for various use cases and provides a significant amount of insights to how the algorithm works in terms of antenna optimization problems. The key focus area of the paper is microstrip antennas. The paper also takes into account problems of different complexities. The authors emphasis that the fitness function is the only connection between the optimization algorithm and the physical problem being considered. A comparative study of the differnet extensions and implementations of the genetic algorithm are also presented.
\item In \cite{circpol_anfis_ga}, A. A. Heidari et al. presented an approach for optimization of circularly polarized antennas with adaptive neuro-fuzzy inference system (ANFIS) and GA. Here, an ANFIS model is trained to estimate the axial ratio and return loss of the antenna from its physical dimensions. The trained ANFIS model is used as a part of the objective function of the optimization problem involving GA. The objective function is defined as a function of the return loss parameter (S11) and axial ratio.
\item B. K. Behera in \cite{fractal_bowtie_ga} presented an approach to optimize a Sierpinski bow-tie (SBT) antenna with GA. The SBT antenna has a fractal geometry with four iterations. The antenna operates at three bands. The performance of the antenna is optimized for each of the resonant frequencies using genetic algorithm. The gain and bandwidth of the antenna are enhanced at all its resonant frequencies. The antenna is also modeled with an equivalent circuit model.
\item An geometry of a rectangular patch antenna is optimized in \cite{txm_opt}. Here, the transmission line model of rectangular patch antenna is used for estimating its resonant frequency, as a part of the fitness function. Multiple design cases are presented in this work, where the length and the width of the patch are tuned with GA to obtain resonance at different frequencies. The length and the width of the feed line are also tuned to optimize the return loss of the antenna.
\item In \cite{patch_feed_ga}, a wide-band high-gain antenna is optimized using genetic algorithm. Here, the impedance of the antenna is calculated through a transmission line analysis. The feed line of the antenna has a tapered section, the width of which is tuned with GA to optimize the input impedance of the antenna. The gain and directivity of the antenna are also maximized because of the optimized impedance matching.
\item Another work where the physical dimensions of a custom microstrip antenna is optimized using GA with transmission line model for the fitness function is reported in \cite{gain_bw_opt_ga}. The topology of the antenna is derived from a rectangular patch with a tapered edge along with slots at the patch and the ground plane. Here, the far-field gain and the operating bandwidth of the antenna are calculated through transmission line model and the same are used in the objective function to maximize these two parameters.
\item \cite{mpa_triband_opt} is another work where a triple band microstrip antenna is optimized with GA. The topology of the antenna comprises of multiple bar like structures. The structure has twelve physical dimensions that are optimized simultaneously with GA to minimize the cost function. The cost function is designed to take into account the sum of the S11 parameter at all the frequency points in all the operating bands of the antenna.
\item A discrete optimization problem for microstrip antennas is presented in \cite{mpa_multifit_ga}. In a discrete optimization problem, the antenna comprises of multiple smaller elements that can either be present or absent. Here, the structure is optimized with four different fitness functions. Each objective function is designed to minimize the return loss, however, a different mathematical formulation is considered in each case. It is shown that different optimization problems turn off different elements of the antenna resulting in four different topologies of the antenna. Each of the resultant structure has different operating frequencies.
\item Planar inverted-F antenna (PIFA) is a popular antenna, once used widely in mobile devices. A discrete optimization of a dual-band PIFA antenna using genetic algorithm is presented in \cite{opt_pifa_ga_2}. The top and the bottom plane of the antenna are considered to be comprised of multiple smaller elements. The physical dimensions of the antenna structure are optimized to minimize the return loss parameter of the antenna. The antenna is designed to operate the ISM bands of 2.4 GHz and 5 GHz.
\item \cite{opt_pifa_ga} is another work where the synthesis of the PIFA as a discrete optimization problem is presented. Here, the antenna is optimized to operate at multiple bands. The objective function is defined to minimize the return loss at all points in the frequency scale. In the work, the fitness function is formulated in such a way that the return loss is minimized at all possible frequencies in the range from 1 GHz to 10 GHz. Although it is not possible to minimize the resonant frequency at all frequencies, an antenna with multiple resonant frequency is obtained.
\item Discrete problems can also be formulated to miniaturize the overall size of the antenna as in \cite{rectpatch_miniaturize_ga}. Here, the original antenna has all the blocks present. It is a multiband antenna. The optimization algorithm restructures the overall topology of the antenna to minimize its size and achieve the desired resonant frequency.
\item In \cite{compCAD4Ant}, a comparison of differential evolution (DE), genetic algorithm based particle swarm optimization (GA-PSO), and ant-colony optimization (ACO) algorithms are compared for antenna optimization problems where the objective is to locate the best feeding position. It is inferred from the reported observation that GA-PSO takes the minimum number of iterations to converge. ACO, on the other hand, takes the minimum time to converge, thanks to its lower computational compexity.
\item The optimization of a broad-band printed monopole antenna is reported in \cite{auto_opt_matlab}. Here, a methodology is presented where the actual antenna is simulated on a full-wave EM solver. The optimization algorithm, on the other hand, runs on Matlab. A visual basic application (VBA) is created for controlling the full-wave EM solver. The matlab code iteratively writes the parameters of the antenna on a file. The file is read by the VBA and feeds the dimensions to the full-wave solver. The result is again written to a file that is read by Matlab. The use of a full-wave solver makes the objective function more accurate and reliable than conventional methods using a transmission line model or some soft-computational model such as neural networks.
\end{itemize}

\subsection{Analysis of Antennas and Electromagnetic Structures with Equivalent Circuit Models}
There has always been attempts to extend the knowhow of circuit theory to estimate the performance of printed antennas. The transmission line model was the earliest attempt in this direction. Although, the accuracy of these models is less, they are good for explaining how the antenna works. The transmission line model and the cavity model for analysis of microstrip antennas were proposed between mid-1970s and early 1980s \cite{txmPhasedArray, txmLinP, txmRect, txmMc, txmLossy, txmPMA, txm_opt, txm_wideband}. More generalized versions of these models were proposed in 1980s that enabled the analysis of antennas with more complex and arbitrary geometrical structures \cite{gtlmThesis, gtlmMath, gtlm2, gtlmAnnular, gtlmSectorCirc, gtlmAnnuRing, gtlmElliptRing}.

With the evolution of modern computers, full-wave electromagnetic solvers started appearing in the scene. With commercially available software tools such as CST studio{\circledR} and Ansys HFSS{\circledR}, it became possible to simulate antennas with any arbitrary shape with high accuracy. In 1990s and 2000s printed antennas with different arbitrary shapes and structures were investigated for various applications \cite{smallPatch0, BandSize0, HPatch1, uslot1, dualBandCircPol, dualBandWLAN, fractal1, slottedUWB, bandnotchCSRR1, bandnotchEBG1, SlottedPatchModel, spiralSlotGnd, GndSRRPatch}.

The analytical modeling of the antennas, although lost its popularity, remained as a tool for understanding and explaining the working of printed antennas. A new inclusion to the list of analytical models in recent years is the equivalent circuit models. RLC circuits are often found useful in modeling the behavior of printed antennas. Equivalent circuits are usually formulated through analytical observation of the antenna. The parameters of the circuit are tuned to match the result of full-wave simulation. The following discussion covers some of the recent works where analytical or equivalent circuit models are used for the analysis of printed antennas.

\begin{itemize}
\item One of the earliest works on designing an equivalent circuit model of printed antennas with considerations for computer aided design (CAD) was proposed by F. Abboud et al. in 1988 \cite{RectEqCkt-der}. The equivalent circuit model proposed in this paper is based on the cavity model and basic circuit theory. The cavity model is used for determining the resonant frequency of the antenna. There are some dynamic permittivity considerations to account for the fringing field effects. Finally, the circuit is composed of resistors, inductors and capacitors.
\item In \cite{mtm_ebg_eqckt2}, A. Liu et al. proposed the equivalent circuit model of an electromagnetic band-gap (EBG) structure. EBG structures are metamaterial unit cells that often are used for improving the electromagnetic performance of antennas. Here, an equivalent circuit model is designed to mimic the frequency response of antennas. This approach is very commonly used in antennas as well.
\item Unlike narrow band antennas, broad band antennas have multiple resonant frequencies very close to one another. Such antennas are modeled as a cascade of multiple narrow-band RLC filter. Such a work was proposed by Y. Kim et al. in \cite{Broadband_EqCkt}. In this work, a rational-circuit model is used for estimating the RLC values of the circuits for each resonant frequency of the antenna. The resistance, reactance and VSWR of the original antenna could be closely approximated with this approach.
\item Bode-Fano integrals are used for finding the limits on the bandwidth of a RLC circuit. In \cite{Fano_BW_Antenna}, the Bode-Fano integral is used for estimating the bandwidth of an antenna. This is a typical example of how the equivalent circuit modeling of antennas help in its further analysis through the knowhow of circuit analysis. The paper shows how the bandwidth of the antenna can be represented by series and parallel inductive or capacitive branches in the equivalent circuit model. This work may be a framework of modeling more complex antenna structures.
\item In \cite{UwbEqCktMethod}, the equivalent circuit model of a UWB antenna is presented. Here also, the antenna is modeled as a cascade structure of RLC bandpass filters. Here, the antenna bandpass filtering elements corresponding to each resonant frequency of the antenna is derived from its Foster canonical form. In this work also, the values of the RLC elements of the equivalent circuit models are selected to mimic the frequency response of the UWB antenna obtained from full-wave simulation.
\item Another work where a broad-band antenna is modeled as cascaded filters is \cite{rectEqCkt}. Here, the equivalent circuit of a rectangular patch and an E-shaped patch is derived from its frequency response. A mathematical formulation is presented for calculating the values of the RLC components of the circuit from its resonant frequency. The result shows a significant agreement between the results obtained from the model and that from the actual antenna.
\item \cite{UwbNotchEqCkt} proposes the equivalent circuit model of a printed UWB antenna with a notch in its resonant frequency. Here, an analytical approach is used for deriving the equivalent circuit model. The antenna structure is divided into simpler artifacts. Each of these artifacts is modeled individually as smaller RLC circuit. These sub-circuits are combined to obtain the equivalent circuit model of the antenna. Although the frequency response of the model is not exactly the same as that of the antenna, such models help in providing more insights to the behavior of the antenna as compared to cascaded filters.
\item Another work where the accuracy of a filter-based equivalent circuit model is enhanced by iteratively tuning it with simulation result is proposed in \cite{UwbPmaEqCkt2}. Here, a matlab code is written to iteratively tune the values of the RLC components of the equivalent circuit to match the result of the full-wave simulation obtained through CST Microwave Studio$^{\circledR}$. It is shown that the proposed approach significantly improves the accuracy of the equivalent circuit model.
\item In \cite{UwbPmaEqCkt3}, a circular printed monopole antenna is loaded with three capacitively loaded line resonators (CLLRs) to obtain resonance at three lower bands. Here, the wide-band printed monopole is modeled as parallel RLC components in series. Each CLLR is also modeled as parallel RLC branches connected in series with the equivalent circuit of the printed monopole. An expression for the input impedance of the antenna is derived from the equivalent circuit model. Although the values of the lumped components were not calculated in this work, it is shown that the input impedance derived from the model resembles that of the actual antenna.
\item Another work where each structural element of a UWB antenna is represented by the a RLC component is reported in \cite{UwbPmaEqCkt4}. Here also, the antenna is modeled as a series of parallel RLC components in series with each other each of which corresponds to a resonant frequency of the antenna. A notch band is introduced in the UWB band by creating a H-slot on the ground plan. The notch band is modeled as another LC tank circuit which is in shunt with the overall equivalent circuit model. This work is a hybrid between analytical equivalent circuit model and filter based equivalent circuit model. Each artifact of the antenna corresponds to a resonant frequency of the antenna, that in terms is modeled as an individual filter element in the equivalent circuit model.
\item The hybrid analytical and filter based equivalent circuit model has been applied to some of the highly complex antenna structures as well. One such example is \cite{ana_eqckt_ex1}. Here an antenna is designed for short range communication that includes several circular rings and vias. The antenna alsi have some arc-shaped elements surrounding its outer circular patch to boost the omnidirectional gain. The antenna is modeled as a combination of multiple RLC components in parallel to account for various artifacts of the antenna. Although, the model is not used to predict the behavior of the antenna, it definitely helps in understanding the working of the antenna.
\item Rational function approximation is one of the earliest approaches to computer aided modeling of printed antennas. In \cite{comp-aided-eqckt}, vector fitting technique is used for rational function approximation of the equivalent circuit modeling of a broad-band antenna. Here, some passivity constraints are imposed on the rational function to ensure that the equivalent circuit model is purely passive. In a second-order rational function approach, the passivity constraints may be directly imposed through quadratic programming and solution can be obtained. For higher order system, however, the particle swarm optimization (PSO) algorithm is used to apply the passivity constraint and obtain the solution. It is emphasized by the author that selecting a suitable optimization algorithm is important depending upon the complexity of the equivalent circuit model and the desired level of accuracy.
\item A combination of an analytical narrow band model and a rational functions based macro-model is proposed in \cite{UwbPmaEqCkt1}. In this work, a narrow band equivalent circuit model of the antenna is derived as a combination of a series and parallel RLC branches. The macro-model is proposed as a rational function of some parameters obtained through data-fitting. In order to simplify the perturbation of the poles of the narrow-band model by the macro-model, the values of the component values are manually adjusted in its SPICE simulation. This approach significantly reduces the computation time for the optimization process without significantly affecting the accuracy of the results.
\item A computer aided formulation of the cavity model of a substrate integrated waveguide (SIW) antenna is presented in \cite{comp-aided-eqckt2}. Here the vector electric field inside the substrate are observed from simulation results. Each resonant frequency of the antenna is modeled as a mode of excitation of the cavity model. Each mode is represented as a tank RLC circuit. The overall circuit is represented as a network of RLC circuits that get excited at different frequencies. The coupling between the cavity and the feed line is modeled with a series resistance and a shunt inductance. It is shown that the equivalent circuit can correctly predict the resonant frequency of the antenna at different modes of excitation.
\item Transmission line model (TLM) of a super wide band (SWB) monopole antenna with three notch bands is proposed in \cite{optEqCkt_ads}. Here also, the antenna is modeled as multiple filtering elements in cascade. The resonant frequencies of the antenna are modeled as band-pass RLC circuit elements whereas the notch bands are modeled as band-stop filter elements. The circuit is further simplified by collecting and shunt R, C and series L components corresponding to the notch filter into a single equivalent shunt R, C and series L component. The component values of the TLM are calculated using quasi-Newton method. The equivalent circuit model provides close approximation of the frequency response of the actual antenna.
\end{itemize}

\subsection{Synthesis and Optimization of Arrays} \label{c1sec_lit_surv}

A phased array antenna is widely used in wireless communication and radar systems. With the evolution of 5G and millimeter-wave communication, a large grid of small printed antennas is becoming popular \cite{5gmmwave, 5gmmwave_fr4}.

A phased array antenna comprises stationary elements excited at different phases to obtain radiation in different directions \cite{phasedArrayHandbook}. Phased arrays have been there for a long time. The first phased array antenna was made in 1955 \cite{phasedArray_russia}. The first printed phased array was reported by Munson et al. in 1974 \cite{txmPhasedArray}. With the evolution of microwave and millimeter-wave communication standards, the use of phased arrays became more common. There has been extensive research on phased array antennas with a significant number of radiating elements for 5G wireless communication \cite{mmarrayRrev}.

Soft-computational optimization algorithms are becoming popular in the synthesis of large planar arrays \cite{arrayTradeoffs}. This section reviews some recent advancements in this direction.

\begin{itemize}
\item In \cite{arraySynth1}, an improved version of the boolean particle swarm optimization (BPSO) with adaptive velocity mutation is used for the synthesis of a non-uniformly spaced linear array of dipole antennas. The spacing between each element of the array and the position of excitation are optimized with the proposed algorithm to improve gain, reduce side lobe level (SLL). The approach is shown to be effective for both broad-side arrays and phased arrays.
\item A planar concentric circular array is thinned using an improved version of differential evolution (DE) called differential evolution with global and local neighborhoods (DEGL) in \cite{arrayThin1}. This is a binary optimization problem which is initialized with a fully populated array. Several different formulations of the optimization problem is presented in the paper each of which is designed to minimize SLL and maximize half power beam-width (HPBW).
\item \cite{nunUniformLinear} shows a comparison of Firefly Algorithm (FA) and PSO for optimization of a non-uniformly spaced linear array. The objective function for the optimization problem is derived from a mathematical formulation of the radiation pattern of the array as a function of its inter-element spacing. It is shown that FA converges faster than PSO in this problem. Further, the result obtained from PSO better matches the pre-defined requirements of the array in terms of beam-width, SLL and null. Similar observation is reported for two arrays designed with varying design goals.
\item An approach for solving the ``Array Thinning Problem'' using genetic algorithm is presented in \cite{thinningGA}. The paper considers dynamic conditions in array thinning problem for both linear arrays and planar arrays. Dynamic thinning problem includes obtaining null at pre-defined angles and rotating the main lobe in different directions. A number of techniques are discussed in the paper to meet the challenges of the dynamic optimization problem including \emph{array symmetry}, \emph{bulk array computation}, \emph{zoning technique}, \emph{thinning Simple Genetic Algorithm}, and \emph{dynamic thinning programmer}.
\item The Teaching Learning Based Optimization (TLBO) algorithm for optimization of a non-uniform linear arrays for SLL reduction is presented in \cite{arraySynth3}. The objective function of the problem is defined as a function of its array factor (AF). The authors have presented the performance of the proposed technique for optimization of arrays with different number of antenna elements. In each case, a significant reduction in the SLL is reported.
\item In \cite{arrayThin2}, a binary spider monkey algorithm (binSMO) is proposed for solving the arrat thinning problem. The proposed algorithm is an extension of the spider monkey optimization (SMO) algorithm. The algorithm is achieved by replacing the basic arithmetic operations in SMO by logical operations such as AND, OR, and XOR. The algorithm is applied for thinning a concentric circular array. The objective function is derived from the mathematical expression of the radiation pattern of the array. The proposed algorithm is reported to show better performance than TLBO and FA in terms of the minimum of SLL and the number of switches turned off.
\item Nondominated sorting genetic algorithm II (NSGA-II), a multi-objective implementation of genetic algorithm, was reported for synthesis of a linear array with reduced SLL and side-band level (SBL) is presented in \cite{arraySynth2}. The objective function is derived from the array factor analysis of the antenna to calculate SLL and SBL as a function of the static excitation and the ``switch-on'' time interval for each element. The performance of the optimized array is evaluated from numerical analysis.
\item In \cite{randomlySpacedArray}, a planar array with randomly spaced elements was presented to obtain low peak side lobe level (PSLL) using a modified GA. Here also, the objective function is derived from the array factor calculation. Each gene in the modified GA represents the inter-element spacing of the antenna, thus, each chromosome represents an iteration of the array structure. It is claimed that this approach can achieve reduced PSLL without affecting the HPBW.
\item A multi-objective approach for self-organizing span array with constraints is presented in \cite{selfOrgOpt}. This paper the objectives are to minimize the number of elements of the array and to reduce the SLL. The primary focus of the paper is to formulate two optimization algorithms inspired by PSO and GA respectively. Finally, another system is developed for finding the optimal solution by combining the results of both algorithms with a minimum Manhattan distance assisted triangle-approximation.
\end{itemize}

\section{Motivation} \label{c1sec_motiv}
After a detailed literature survey, it is observed that soft-computational optimization techniques are attracting attention of researchers worldwide in the fields of antenna design and array synthesis. Three different categories of research are reviewed in the preceding section and some research gaps are identified in each category.

\begin{itemize}
\item \textbf{Design or Synthesis of Printed Antenna Elements:} Various soft-computational techniques have been used for designing antenna elements. The optimization is usually an iterative process where the value of the objective function is calculated iteratively for different combinations of the input variables. Eventually, the algorithm converges to a solution that either minimizes or maximizes the objective function. For antenna design problems, it is important to simulate the antenna. Usually, the optimization algorithm runs on a computational tool such as Matlab$^{\circledR}$. whereas the antenna simulation runs on a full-wave electromagnetic solver such as Ansys HFSS, CST Studio, etc. For simple antenna structures, it is possible to calculate the antenna output using analytical models, but the accuracy of such models is low as the complexity of antenna structures become high.
\item \textbf{Analysis of Antennas and Electromagnetic Structures with Equivalent Circuit Models:} Analytical modeling of antennas have remained a popular area of research since the beginning of antenna research. Traditionally, these models were used for calculating the performance of antennas. With the evolution of soft-computational optimization techniques, computer aided derivation of analytical models is becoming popular. Most of the time, the antenna is modeled as a series of cascade filter structure. Such models, although can accurately describe the frequency response of an antenna, does not provide much insights into the working of the antenna.
\item \textbf{Synthesis and Optimization of Arrays:} With the evolution of 5G and millimeter wave communication, synthesis of optimized planar arrays is becoming a popular area of research. Although there have been a significant research on side lobe reduction, there is still a need for formulating techniques that reduces the number of elements in a planar array without significantly altering the radiation pattern of the array at most of its scan angles. The optimization should consider more realistic models for antenna elements. There are better tools for analyzing the radiation pattern of phased array systems that can serve better for calculating the objective functions. All these areas need exploration.
\end{itemize}

%\section{Objectives}
%From the detailed literature survey presented in Section \ref{c1sec_lit_surv}, it is to be noted that the optimization of printed antennas is mostly related to finding the best algorithm, and

\section{Problem Formulations} \label{c1sec_prob}
With the extensive literature survey is presented With the extensive literature review and the research gaps identified in sections \ref{c1sec_lit_surv}, and \ref{c1sec_motiv} respectively, the following problems are identified in each categories.

\begin{itemize}
\item \textbf{Design or Synthesis of Printed Antenna Elements:} There have been a significant amount of works on printed antenna optimization in the recent years. However, it is observed that the number of works reported on printed slot antennas is less. Slot antennas typically have an omnidirectional radiation pattern and a wider bandwidth than microstrip antennas. These features make slot antennas a suitable candidate for indoor short range communication. Thus, the first problem identified is to optimize a printed slot antenna using genetic algorithm.

    It is to be noted that for optimizing an antenna using soft-computational techniques, an interface between the computational platform and the full wave simulators is to be created. This adds up to the computational cost of the overall process. Simplifying this process is another goal that is to be addressed while optimizing the slot antenna.
\item \textbf{Analysis of Antennas and Electromagnetic Structures with Equivalent Circuit Models:} Developing equivalent circuit models of antennas has remained a popular area of research for decades. Computer aided synthesis of the equivalent circuit model involves estimating the values of the lumped elements of the equivalent circuit from the frequency response of the antenna obtained from full-wave simulation. In most of the reported cases, the equivalent circuit model is a set of filters that corresponds to the resonant frequency of the antenna. Such circuits usually don't provide much insights into the working of the antennas. The second problem identified is to design an analytical equivalent circuit of a multi-layer antenna structure that perfectly illustrates the working of the antenna. The values of the lumped elements of the equivalent circuit is to be estimated using soft-computational techniques. Further, an attempt will be made to correlate the frequency response of the S11 parameter and the surface current distribution of the antenna from the equivalent circuit model.
\item \textbf{Synthesis and Optimization of Arrays:} Array thinning is one of the most popular area of research in array optimization. In array thinning, the most common problem is the reduction of SLL. The third problem is to create an improved technique for array thinning. The idea is to reduce the number of elements in a uniform rectangular array (URA) by 50\% while maintaining its radiation pattern, side-lobe level and beam-width close to the original URA.

    Another aspect of the array optimization problem is to use more reliable models of the array and the antenna elements so that a higher accuracy of the optimized antenna can be guaranteed. Traditional analytical methods used for estimating the radiation patterns of an array will be compared with soft-computational modeling tools such as the artificial neural network (ANN) to estimate which approach provides higher accuracy.
\end{itemize}

\section{Methodology} \label{c1sec_method}
Soft-computational optimizations iterate through the solution space to arrive at a point of minimum or maximum value of the objective function \cite{softCompBook}. This process is illustrated in Fig. \ref{fch-soft-flow}. For antenna optimization, the variables to be optimizes are the physical dimensions of the antenna. The objective function is typically the difference between the desired value of the electromagnetic parameter and the actual value of these parameters. Depending upon the complexity of the problem and the nature of the optimization algorithm, the mathematical expression to find the difference may vary. It can be evident that the process requires the electromagnetic parameters of the antenna to be evaluated iteratively in a loop. This can be performed using a high-fidelity model such as Finite Difference in Time Domain (FDTD), Method of Moments (MoM), Integral Equation (IE) analysis, or Finite Elements Method (FEM). Several software applications available that implement these algorithms are highly reliable and are suitable for antennas of almost any shape and topology. However, these algorithms are computationally expensive and time-consuming. The analysis of a single antenna can take several minutes using these techniques depending upon the complexity of the structure.

\begin{figure}
\includegraphics[width=0.6\linewidth]{soft-flow1.eps}
\caption{Flow diagram of a soft-computational approach to antenna design}\label{fch-soft-flow}
\end{figure}

\section{Contributions} \label{c1sec_contrib}
The chapter-wise contents and summaries of the work are outlined below-
\begin{enumerate}
\item \textbf{Chapter 1} gives a review of the background. Literatures related to our work, motivation and objective of the work including problems identified are included in this chapter. The literature survey has been divided into three distinct parts and the motivation and problem formulations are also classified accordingly.
\item \textbf{Chapter 2} includes the basic theory of antenna design, array design, optimization strategies, and types of optimization problems. The entire research work is based on the theoretical construct of this chapter.
\item In \textbf{Chapter 3}, the original research work on optimization of the feeding position of a slot antenna using genetic algorithm is presented. Here, a technique for implementing genetic algorithm within Ansys AEDT is presented which significantly reduces the computational overhead for antenna optimization.
\item \textbf{Chapter 4} is another original research where an antenna is designed that has an omnidirectional radiation pattern at the 2.4 GHz ISM band and a directional radiation pattern at the C-band of 5G wireless communication. An equivalent circuit model of the antenna is presented. The values of the lumped elements of the equivalent circuit model are estimated using vector fitting and genetic algorithm.
\item In \textbf{Chapter 5}, an array thinning problem is addressed using particle swarm optimization technique. Here, the number of elements of a large grid of printed antennas is reduced by 50\% without significantly affecting its performance in terms of SLL and beam-width.
\item \textbf{Chapter 6} concludes the thesis. It summarizes the present research work. Findings of the present work, limitations of the present research and the scope for further improvements is described in this chapter.
\end{enumerate}
\section{Organization of the Thesis} \label{c1sec_organiz}
This section outlines the description of the thesis. The \textbf{Chapter 1}, includes the background of the work in Section \ref{c1sec_background}. A detailed literature survey in Section \ref{c1sec_litserv}. Motivation, and Problem formulation are presented in Sections \ref{c1sec_motiv} and \ref{c1sec_prob} respectively. Section \ref{c1sec_contrib} includes the contribution of each chapter. The organization of the thesis is described is Section \ref{c1sec_organiz}. The chapter is concluded in Section \ref{c1sec_concl}.

\section{Conclusion} \label{c1sec_concl}
The chapter starts with a background of the overall work. A detailed literature survey is presented for three different classes of research trends related to the soft-computational optimization of printed antennas and arrays. The motivations and problem formulation for category are included in this chapter. Finally, the contributions of each chapter along with the organization of the whole thesis is presented. 