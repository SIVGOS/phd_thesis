\begin{abstract}
The techniques for the design and analysis of printed antennas and arrays have evolved through various phases in the last few years. In the initial years, the analysis of printed antennas was mostly done through analytical models. These models, despite being useful in explaining the behavior of the antennas, are not high in accuracy and precision. Modern full-wave electromagnetic solvers, on the other hand, are highly reliable and capable of analyzing antennas with any arbitrary geometry accurately. The computational cost of these models, however, is significantly high. In recent years, soft-computational optimization algorithms have earned attention among researchers for solving complex problems. Such algorithms iterate through the solution space of a given problem to find an optimal solution subject to given constraints.

Research works pertaining to the design and optimization of printed antennas and arrays may be classified into three categories --- (a) \emph{setting the physical dimensions of a printed antenna}, (b) \emph{analytical modeling of printed antennas for providing insights to the working of the antennas}, and (c) \emph{finding optimal positions of the elements of a printed antenna array}. With the availability of full-wave electromagnetic solvers, it is possible to predict the electromagnetic parameters of an antenna such as resonant frequency, gain, bandwidth, etc. with high accuracy. When such models are used for iteratively computing the performance of the antenna during an optimization process, the resultant antenna is ready for real-world deployment.

The first part of the work presents a technique for optimization of a printed slot antenna using genetic algorithm (GA). The genetic algorithm is implemented within the script interface of Ansys Electronic Desktop (AEDT$^{\circledR}$) to speed up the optimization process. A typical setup for optimization of antennas using GA involves the implementation of the optimization algorithm in a numeric computational platform such as Matlab and solving the antenna structure in a full-wave electromagnetic solver. The proposed approach eliminates running two resource-intensive software tools on a computer and establishing a link between them, thus significantly reducing the computational cost. The optimized antenna is fabricated on a copper-clad PCB board and the performance of the fabricated antenna is measured to validate the accuracy of the optimization technique.

In the second part of the work, the equivalent circuit model of a multi-layer, dual-band printed antenna is derived. The antenna has an omnidirectional radiation pattern at the 2.4 GHz ISM band and a directional radiation pattern between 3.5 GHz to 4.2 GHz. The designed antenna is studied with an analytical equivalent circuit model. The circuit parameters are estimated with a vector fitting method and with the genetic algorithm. The behavior of the antenna is explained with the equivalent circuit model. The designed antenna is fabricated and measurement results are found to be in agreement with the simulation results. The proposed antenna has possible applications in 5G and IoT implementations.

Computer-aided synthesis of a sparse array is a popular area of research worldwide for the application in radar and wireless communication. The trend is observing new heights with the launch of 5G millimeter-wave wireless communication. A sparse array has a fewer number of elements than a conventional antenna array. In the third part of the work, a sparse array is synthesized from a $16\times 16$ uniform rectangular array (URA). The synthesis includes an artificial neural network (ANN) model for estimation of the excitation weights of the URA for a given scan-angle. The weights of the sparse array are computed by the Hadamard product of the weight matrix of the URA with a binary matrix that is obtained using particle swarm optimization (PSO). The objective function of the optimization problem is formulated to ensure that the PSLL is minimized for multiple scan-angles. It is shown from an experimental analysis that apart from minimizing the PSLL, the proposed approach yields a narrower beam width than the original URA.
\end{abstract} 